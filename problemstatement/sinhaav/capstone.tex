\documentclass[onecolumn, draftclsnofoot,10pt, compsoc]{IEEEtran}
\usepackage{graphicx}
\usepackage{url}
\usepackage{setspace}

\usepackage{geometry}
\geometry{textheight=9.5in, textwidth=7in}

% 1. Fill in these details
\def \GroupMemberOne{			Aviral Sinha, Tyler Jones, Samuel Cooney}
%\def \GroupMemberTwo{			Tyler Jones}
%\def \GroupMemberThree{			Samuel Cooney}
\def \CapstoneTeamNumber{		68}
\def \CapstoneProjectName{		Investment Performance Mobile App}
\def \CapstoneSponsorCompany{	HedgeServ}
\def \CapstoneSponsorPerson{	Brice Lemke}

% 2. Uncomment the appropriate line below so that the document type works
\def \DocType{		Problem Statement
				%Requirements Document
				%Technology Review
				%Design Document
				%Progress Report
				}
			
\newcommand{\NameSigPair}[1]{\par
\makebox[2.75in][r]{#1} \hfil 	\makebox[3.25in]{\makebox[2.25in]{\hrulefill} \hfill		\makebox[.75in]{\hrulefill}}
\par\vspace{-12pt} \textit{\tiny\noindent
\makebox[2.75in]{} \hfil		\makebox[3.25in]{\makebox[2.25in][r]{Signature} \hfill	\makebox[.75in][r]{Date}}}}
% 3. If the document is not to be signed, uncomment the RENEWcommand below
\renewcommand{\NameSigPair}[1]{#1}

%%%%%%%%%%%%%%%%%%%%%%%%%%%%%%%%%%%%%%%
\begin{document}
\begin{titlepage}
    \pagenumbering{gobble}
    \begin{singlespace}
    	%\includegraphics[height=4cm]{coe_v_spot1}
        \hfill 
        % 4. If you have a logo, use this includegraphics command to put it on the coversheet.
        %\includegraphics[height=4cm]{CompanyLogo}   
        \par\vspace{.2in}
        \centering
        \scshape{
            \huge CS Capstone \DocType \par
            {\large\today}\par
			{\large CS461 Fall 2017}\par
            \vspace{.5in}
            \textbf{\Huge\CapstoneProjectName}\par
            \vfill
            {\large Prepared for}\par
            \Huge \CapstoneSponsorCompany\par
            \vspace{5pt}
            {\Large\NameSigPair{\CapstoneSponsorPerson}\par}
            {\large Prepared by }\par
            Group\CapstoneTeamNumber\par
            % 5. comment out the line below this one if you do not wish to name your team
            \CapstoneTeamName\par 
            \vspace{5pt}
            {\Large
                \NameSigPair{\GroupMemberOne}\par
                %\NameSigPair{\GroupMemberTwo}\par
                %\NameSigPair{\GroupMemberThree}\par
            }
            \vspace{20pt}
        }
        \begin{abstract}
        % 6. Fill in your abstract    
        	This document outlines the project that my group members and I will be working on. Over the course of this next year we will be working with HedgeServ on a investment performance project that will help mobile users with their portfolio. This project proposal will be entailing a definition and description of the problem as well as our proposed solution. As one of the industry leaders in independent fund administration, the project is something that will help HedgeServ continue its ability to serve hedgefunds in various strategies and continue their 100 percent retention rate among clients. By understanding both finance and software engineering we plan on releasing a product at an extremely high level of quality that exceeds the standards set by HedgeServ.
        \end{abstract}     
    \end{singlespace}
\end{titlepage}
\newpage
\pagenumbering{arabic}
\tableofcontents
% 7. uncomment this (if applicable). Consider adding a page break.
%\listoffigures
%\listoftables
\clearpage

% 8. now you write!
\section{Definition and Description of Problem}

HedgeServ has tasked us with the creation of a mobile application that lets users monitor the performance of their investments at a portfolio level and then look into specific investments. HedgeServ currently has relationships with almost 200 clients including hedge funds, private equity funds, and other insitutional investment managers. The company also supports various strategies used in the industry that include but aren't limited to long/short equity, fixed income, distressed mergers and acquisitions, and emerging markets. While the company is headquarted in New York City we'd working mostly with the development team that is based in Portland. By working with Brice Lemke and other people within the HedgeServ team we will be using Xamarin and C#/F# to develop financial application that can be used with both android and iOS mobile devices. We will be provided with financial data from HedgeServ that will then show holdings, gains and losses, transaction history, and then charts and graphs. 

\section{Proposed Solution}

 We've only had one preliminary meeting with Brice, so we are still not sure on the specifics. We'd using a mix of .NET and C# that will prices of stocks from publicly available sources and then show calculations of investment performance. We would also build databases that would contain the prices for us to pull and analyze, then use the list of transactions and use it as test data sets.  Our real time prices won't be provided and we would have to develop a way to update it consistently. We'd also have to keep scalability in mind as it will be used by a couple thousand users. Then for the user interface we haven't been provided with a guideline that is standard throughout HedgeServe but we are looking into other financial services applications for inspiration, specifically Robinhood and Merrill Edge which are both mobile brokers with a very clean and simple interface. Over the course of the next 3 terms our action plan will be to start planning and understanding of our problem at hand and then start to figure out the specifics of what will be included in our application then throughout winter term begin and complete most of the development for our application and then in spring term complete our application and prepare it for the senior design presentations. By properly following a time table with set dates on when we want to complete assignments and efficiently allocating work to each other the project for this will be both a great learning expierence for all of us and prepare us for future careers in computer science. As for performance metrics we have a rough idea that the product will be used atleast internally as a part of quality assurance testing so we need to be able to have an application that does its intended function as well as possible. Our preliminary meeting with Brice we briefly discussed what he wanted from us in terms of the product and other than what was listed in the project details we should really focus on making it as usable as possible because with the route financial services is taking many funds are now turning to applications for help with the portfolio management as it can give better real time analytics on various stocks and give both a detailed and big picture view of what the implications of certain decisions can be. The databases we'll be implementing will follow most of the guidelines set by the company in terms of what entities will have relationships and how they should be interacting with each other. Also we'd have to make sure that the scalability is up to par with what HedgeServ wants and we'd verify if it i can handle the intended userbase and then some through testing. Implementing test driven development is also a key solution in getting the best possible result for our project because through this method we will be able to constantly make sure our product is at its best quality because we'd be testing most scenarios that the application will face.  
	
\end{document}





