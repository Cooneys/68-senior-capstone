\documentclass[onecolumn, draftclsnofoot,10pt, compsoc]{IEEEtran}
\usepackage{graphicx}
\usepackage{url}
\usepackage{setspace}

\usepackage{geometry}
\geometry{textheight=9.5in, textwidth=7in}

% 1. Fill in these details
\def \CapstoneTeamName{		The Cleverly Named Team}
\def \CapstoneTeamNumber{		68}
\def \GroupMemberOne{			Tyler Jones}
\def \GroupMemberTwo{			Avi Sinha}
\def \GroupMemberThree{			Sam Cooney}
\def \CapstoneProjectName{		Investment Performance Mobile Application}
\def \CapstoneSponsorCompany{	HedgServe}
\def \CapstoneSponsorPerson{		Edison Tsai}

% 2. Uncomment the appropriate line below so that the document type works
\def \DocType{		%Problem Statement
				%Requirements Document
				%Technology Review
				Design Document
				%Progress Report
				}
			
\newcommand{\NameSigPair}[1]{\par
\makebox[2.75in][r]{#1} \hfil 	\makebox[3.25in]{\makebox[2.25in]{\hrulefill} \hfill		\makebox[.75in]{\hrulefill}}
\par\vspace{-12pt} \textit{\tiny\noindent
\makebox[2.75in]{} \hfil		\makebox[3.25in]{\makebox[2.25in][r]{Signature} \hfill	\makebox[.75in][r]{Date}}}}
% 3. If the document is not to be signed, uncomment the RENEWcommand below
%\renewcommand{\NameSigPair}[1]{#1}

%%%%%%%%%%%%%%%%%%%%%%%%%%%%%%%%%%%%%%%
\begin{document}
\begin{titlepage}
    \pagenumbering{gobble}
    \begin{singlespace}
    	\includegraphics[height=4cm]{coe_v_spot1}
        \hfill 
        % 4. If you have a logo, use this includegraphics command to put it on the coversheet.
        %\includegraphics[height=4cm]{CompanyLogo}   
        \par\vspace{.2in}
        \centering
        \scshape{
            \huge CS Capstone \DocType \par
            {\large\today}\par
            \vspace{.5in}
            \textbf{\Huge\CapstoneProjectName}\par
            \vfill
            {\large Prepared for}\par
            \Huge \CapstoneSponsorCompany\par
            \vspace{5pt}
            {\Large\NameSigPair{\CapstoneSponsorPerson}\par}
            {\large Prepared by }\par
            Group\CapstoneTeamNumber\par
            % 5. comment out the line below this one if you do not wish to name your team
            \CapstoneTeamName\par 
            \vspace{5pt}
            {\Large
                \NameSigPair{\GroupMemberOne}\par
                \NameSigPair{\GroupMemberTwo}\par
                \NameSigPair{\GroupMemberThree}\par
            }
            \vspace{20pt}
        }
        \begin{abstract}
        % 6. Fill in your abstract    
        The purpose of this document is to go over design decisions for the Investment Performance Mobile Application. This document serves to clarify the intended 
	audience of the application as well as elaborate upon and discuss the design decisions from various viewpoints. 
        \end{abstract}     
    \end{singlespace}
\end{titlepage}
\newpage
\pagenumbering{arabic}
\tableofcontents
% 7. uncomment this (if applicable). Consider adding a page break.
%\listoffigures
%\listoftables
\clearpage

% 8. now you write!
\section{Introduction}
\begin{table}[H]
			\caption{Table 1 - Summary of Design Viewpoints}
			\begin{center}
				\begin{tabular}{| p{0.3\linewidth} | p{0.3\linewidth} | p{0.3\linewidth} | }
					\hline
					 \textbf{Design Viewpoint} & \textbf{Design Concerns} & \textbf{Example Design Languages} \\ [0.5ex]
					%heading
					\hline
					Context(5.2)  & Create a schedule for remaining documents and finish project proposal & Organize a group meeting and client meeting  \\
					\hline
					 Successfully integrated with our new client at a reasonable pace & Edit problem statement and plan the requirements & Go over feedback on problem statement and hold a group meeting \\
					\hline
					 Submitted Problem Statement with little difficulty & Formatting and additional content need to be updated & Go over feedback on the requirements document with the group \\
					\hline
					Requirements document properly revised and resubmitted after client dropoff & Finish revision of project proposal and begin technology review planning & Have a group meeting to work on revisions and plan for the upcoming document\\
					\hline
					Technology review extension received and still finished on time & Finish and submit technology review & Hold a group meeting early in the week to get a head start on the technology review\\
					\hline
					Meetings with new client went smoothly & Begin working on the design document & Meet with the client to discuss design decisions in preparation for the design document  \\
					\hline
					Communicated with Kevin, Kirsten, and Andrew often and early about client situation & Finish design document and start progress report & Have group meeting to finish design \\
					\hline
					& & \\
					\hline
					\hline
					\hline

				\end{tabular}
			\end{center}
			\end{table}


\section{Context viewpoint}
	Our financial mobile application aims to supplement users' investment experience by providing insight into the quality of investments from a individual and portfolio level. 
	Relevant stakeholders include the developers of this application, namely Tyler Jones, Avi Sinha, and Sam Cooney, as well as HedgServe employees namely Edison Tsai and Ronald Olshausen.

	From a "black box" perspective, the application will allows users to enter an investment, including type of asset class, either stock or stock option, company name and ticker symbol, 
	and amount invested combined with shares purchased. The user will then have proper pricing data displayed for the investment for a given time period, and the portfolio and individual 
	investments will display the performance of the user's investments based on such pricing.

\subsection{Design concerns}
	The scope and use of this application is to be a learning tool, both for the relevant developers, as well as any future users. Typically, in an enterprise environment, many audits and 
	checkpoints must be performed on the application in order to verify it as a reliable source of financial advice. Due to the time span and scope of this application, such regulation and 
	checkpoints are not going to be made, and as such, this application is not intended for the purpose of either handling real finances, nor to be used to make fully informed financial decisions. 
	Any relevant statistics displayed for the user based on their current investments should be investigated more in depth by said user.

\subsection{Design Elements}
	Design entities: Actors - A MySQL database will be interacting directly with the front end of the application, and thus the user. When a user enters a new investment into the application 
	through the UI, said entry is stored in the MySQL database and can later be retrieved as the user needs. Moreover, API (Quandl, Avi?) calls will be made that allow the users stored investment 
	information to have relevant pricing data displayed.

\section{Composition viewpoint}
\subsection{Design Elements}
	Design Entities: The investment performance mobile application can be broken down in 3 primary components/entities. These components are the user interface, the database, and the API.

\subsection{Design Relationships}
	The database, and chosen financial data API both will work with the user interface. The database will contain the financial information that was entered by the user, and the API will 
	supplement such data with relevant pricing info in order to give performance metrics. Both the API and database will be displayed on the front end of the application, and will appear
	to the user in a desirable fashion through the interface.

\subsubsection{Function Attribute: API}
\subsubsection{Function Attribute: User Interface}
\subsubsection{Function Attribute: Database}
	The database will hold the financial data that was entered by the user. This database will be a relational database that is hosted by Oregon State University. Such a choice was 
	made due to the financial advantage that using this database provides. Every student at the university is given 1 free database, and for this reason it was chosen. The database 
	management system that will be used is MySQL.

	The database will contain not only financial data, but will also contain login information, and the number of portfolios that belong to a specified user. A user can share a portfolio 
	with another user, as well as have an individual portfolio. In the case of joint custody, the two users will both be able to retrieve and access the portfolio from their interface. 
	A relational database, such as the one that will be implemented, is perfect for this purpose.



\section{Logical viewpoint}
	I dont think this section is relevant. Need Sam and Avi approval to delete.
\section{Dependency viewpoint}
	This section seems redundant with section 3.2 (design relationships) . Need Sam and Avi approval to delete. Perhaps this can be more detailed than 3.2?
\section{Information viewpoint}
	As stated previously, the users financial data will be stored in a MySQL database. This section intends on going into more detail about the implementation and expectation of stored 
	data based on what the user will be able to enter from within the user interface.

	At this point in time, the user will be able to provide the following to the application, and will contain two primary types of tables:

\subsection{User Table}
\begin{itemize}
    \item Login information that is encrypted. Encryption will likely be done with either SHA1 or SHA256. 
    \item Portfolios held. This will be stored as an integer.
    \item Names of portfolios. Portfolios named will contain a foreign key relating to the proper portfolio within the portfolio table.
\end{itemize}
\subsection{Portfolio Table}
\begin{itemize}
    \item Owners of the portfolio. Can be owned by multiple users, or one user.
    \item Total amount invested in USD. This will be stored as a floating point number.
    \item Amount invested per company
    \item Number of shares purchased based on amount invested. This will be stored as a floating point number.
    \item Types of investments, either a stock or stock option. This can be stored as either a string, or because only two asset classes are supported, perhaps a binary value to represent the type (0 or 1).
    \item Ticker Symbol of company invested. This will be stored as a string.

\end{itemize}

\section{Patterns use viewpoint}
I dont think this section is relevant. Need Sam and Avi approval to delete.
\section{Interface viewpoint}
Relevant! how do we ensure from the interface that the data is there?

\section{Structure viewpoint}
I dont think this section is relevant. Need Sam and Avi approval to delete.
\section{Interaction viewpoint}
This section seems redundant with section 3.2 (design relationships) . Need Sam and Avi approval to delete.
\section{State Dynamics viewpoint}
I dont think this section is relevant. Need Sam and Avi approval to delete.
\section{Algorithm viewpoint}

\section{Resource viewpoint}

I dont think this section is relevant. Need Sam and Avi approval to delete.

\end{document}
