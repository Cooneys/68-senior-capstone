\documentclass[letterpaper,10pt,titlepage,journal,compsoc,draftclsnofoot,onecolumn]{IEEEtran}
\linespread{1}
\newcommand\tab[1][1cm]{\hspace*{#1}}
\usepackage{graphicx}                                        
\usepackage{amssymb}                                         
\usepackage{amsmath}                                         
\usepackage{amsthm}                                          

\usepackage{alltt}                                           
\usepackage{float}
\usepackage{color}
\usepackage{url}
\usepackage{listings}

\usepackage{balance}
\usepackage[TABBOTCAP, tight]{subfigure}
\usepackage{enumitem}
\usepackage{pstricks, pst-node}

\usepackage{geometry}
\usepackage{titling}
\geometry{textheight=8.5in, textwidth=6in}



\newcommand{\cblue}[1]{{\color{blue}#1}}

\usepackage{hyperref}
\usepackage{geometry}

\def\name{Garrett Amidon}


%% The following metadata will show up in the PDF properties
\hypersetup{
  urlcolor = black,
  pdfauthor = {\name},
  pdfkeywords = {cs461 ''Senior Capstone''},
  pdftitle = {CS 461 Senior Capstone: Progress Report},
  pdfsubject = {CS461 Senior Capstone},
  pdfpagemode = UseNone
}

\title{Investment Performance Mobile App: Fall Term Progress Report}
\author{Tyler Jones, Avi Sinha, Sam Cooney}

\begin{document}
\begin{titlingpage}
    \maketitle
	\centering{}
    \begin{abstract}
        
     The purpose of the following document is two-fold. The primary purpose is to serve as a means of reflection upon the term, and review what happened and when. The second purpose is to look-back and consider the problems and issues that we faced this term, either as a group or individual, and how to improve on those problems as we progress into development in the coming term.
        
    \end{abstract}
\end{titlingpage}

\newpage

\tableofcontents{}

\newpage

\section{Purpose and Goals}

\tab Our project is an investment performance mobile application. Our main goal with this application is to provide users with a way to enter and track the progress of their investments, so they can gain unique financial insight about the quality of said investments. Typically, through a brokerage, for example, users may only see what investments they have, and they are not always given a composite performance, as is the goal with our project. We aim to give the user a metric for both their entire investment portfolio, as well as the ability to drill down into individual investments for a more localized look.

\section{Weekly Progress}

    \subsection{Week 1}
    During week 1, we looked over and gave our first 3 choices for projects to Kevin. This project was my first choice, as well as Avi's and Sam's
    \subsection{Week 2}
    This week, we had our first meeting with Brice Lemke. We introduced ourselves and presented him with our backgrounds. He gave us a very general overview of what the application would entail which was already basically within the project description. Additionally, we set up a weekly meeting time. We also set up a means of communication through an IM app called Slack.
    
    \subsection{Week 3}
    This week we received our first email from our TA, Andrew. We set up a weekly meeting with him. We also reviewed the rough drafts of our problem statement in class.
    
    
    \subsection{Week 4}
    This week we submitted and finalized our problem statement with our client. We further revised and understood more nuance about the expectations and requirements of the project.

    
    \subsection{Week 5}
       This week we submitted our rough draft for our requirements document, we also met with our TA for the first time this week, Andrew, after it requiring a period of time do agree on a time slot with him 
    
    \subsection{Week 6}
        This week we discovered and we had it shared with us that our client will no longer be at HedgServe. We were supposed to supply our client with our requirements document by Friday, and he didn't respond on any of the days we reached out. We had an email sent to us by one of his coworkers stating that he was no longer working there and we would have to begin working with someone else. 
 
        We submitted our "final" requirements document unsigned to our TA, and will need to have it signed at a later date 
        
        We have plans to meet at our new time with our new client on Wednesdays at 12-1. We will bring him/her up to speed as soon as possible in order to keep the project moving 
            
    
    \subsection{Week 7}
        This week we had our first meeting with our new client, Edison Tsai, and Ronald O. Ronald's vision for our project differed greatly from Brice's so moving forward we are going to have to more or less go back to the drawing board and start from scratch on our requirements document, 
 
        We updated Andrew, Kiersten, and Kevin all individually and informed them of the situation. Going forward we have loose deadlines on some of the documents, and can request extensions if needed. 
 
        This coming week we plan on redoing our requirements document and submitting that to Edison/Ronald for review. 
    
    \subsection{Week 8}
        This week we met with our new client for the second time and got a much better idea of what their vision is for the new project. It was very reassuring, and we received enough information to fully revise and resubmit our requirements document to our client for approval. Next week, we have our tech review due, but Kirsten gave us an extension. I plan on submitting the tech review after the break. 

            
    \subsection{Week 9}
        This week we submitted our new requirements document and discussed it with our client in our weekly meeting. Not much was done or shared this week, and everyone in our group is going different places for thanksgiving and going to write and submit the tech review this weekend/week. 

            
            
    \subsection{Week 10}
        This week, I finally got my Tech Review in after receiving an extension from Kirsten. Additionally, we began discussing the design document and received an extension on that until next Friday 12/8. We also discussed our meetings over winter break with our client, and we agreed that there wasn't a need to meet during our time off and we would reconvene in January. We also got feedback from Andrew about our requirements document, and he shared with us that our biggest flaw was missing IEEE standards, and that we need to make sure to meet them when he grades the design document. 

\section{Current Status}

As of this date, a significant amount of documentation and research has been done about the project. Our problem statement, requirements document, technology review, and design document will all serve us well going forward into Winter term. Each one has a different role, and describes the different pieces of the project in greater detail. The problem statement served to further clarify what is going to be done, and what exactly the end product will do and be. The requirements document stated more rigorously what our goals for the project are, and what metrics we will use to identify if we succeeded in such goals, while the technology review helped explore the various options available to us that we will use to achieve them. The design document further elaborates on how such technologies will be related and connected in order to make sure our requirements are met to a satisfactory degree.

\section{Problems}
We have certainly had our fair share of problems this term. The biggest of said problems is the fact that our original HedgServe POC dropped out for reasons unknown to us. Brice Lemke was replaced with Edison Tsai as our primary representative, and his vision for what the application will do and be is much more condensed and focused in nature than Brice's. Our previous concept of the app covered a host of different investment types. This includes stocks, stock options, cryptocurrency, and property investment just to name a few. Our new concept will solely focus on stocks and stock options, which we think was a necessary change, but was cumbersome to go back and "rethink" the app nonetheless. This client switching issue was mostly a problem for us because it set us back roughly 5 weeks. Thankfully, Kirsten and Kevin were very understanding of the situation and didn't force us to go back and revise our problem statement. However, we did need to revise our requirements document and change it a decent amount due to the vision of the application being substantially different from what our first draft entailed. 

Other than problems that are out of our control, I would say we haven't had many large scale issues. One issue we did have was with our requirements document. Our TA, Andrew, had to share with us the importance of industry standard formatting for this course. Our requirements document was drafted based on our vision of what would be appropriate, however, there is a fairly rigorous IEEE standard available to us that we were expected to follow and didn't due to us not being aware of its importance. On that note, I think one thing we can improve on going forward is standards. Not only in using them, but asking questions about them and reading the listed requirements for each document as we progress. IEEE documentation, for me, is very difficult to understand, and without examples, is hard to know if it is being done correctly. Many times even the examples provided by IEEE documentation are too large scale or broad to really effectively draw any sort of meaningful insight out of them. One more place I think we could do better in is asking questions. This project is very financial in nature. Obviously, none of us are finance majors, so asking questions and making sure that we are understanding the terminology correctly is a necessity if we are going to succeed going forward.


\section{Retrospective}

\begin{table}[H]
			\caption{Retrospective: Fall 2016}
			\begin{center}
				\begin{tabular}{| p{0.3\linewidth} | p{0.3\linewidth} | p{0.3\linewidth} | }
					\hline
					 \textbf{Positives} & \textbf{Deltas} & \textbf{Actions} \\ [0.5ex]
					%heading
					\hline
					Were able to recover from client uproot, and got Fall term work done within Fall term  & Create a schedule for remaining documents and finish project proposal & Organize a group meeting and client meeting  \\
					\hline
					 Successfully integrated with our new client at a reasonable pace & Edit problem statement and plan the requirements & Go over feedback on problem statement and hold a group meeting \\
					\hline
					 Submitted Problem Statement with little difficulty & Formatting and additional content need to be updated & Go over feedback on the requirements document with the group \\
					\hline
					Requirements document properly revised and resubmitted after client dropoff & Finish revision of project proposal and begin technology review planning & Have a group meeting to work on revisions and plan for the upcoming document\\
					\hline
					Technology review extension received and still finished on time & Finish and submit technology review & Hold a group meeting early in the week to get a head start on the technology review\\
					\hline
					Meetings with new client went smoothly & Begin working on the design document & Meet with the client to discuss design decisions in preparation for the design document  \\
					\hline
					Communicated with Kevin, Kirsten, and Andrew often and early about client situation & Finish design document and start progress report & Have group meeting to finish design \\
					\hline
					Communication between me, Avi and Sam has been good. Each of us pull our own weight & Complete progress report and start recording presentation & Organize multiple group meetings for this and next week to complete all work \\
					\hline

				\end{tabular}
			\end{center}
			\end{table}


\end{document}