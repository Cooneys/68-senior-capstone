\documentclass{article}
\usepackage[utf8]{inputenc}
\usepackage{longtable}

\title{Final Report - Tyler's Section}
\author{jonesty }
\date{June 2018}

\begin{document}

\maketitle

\section{Introduction}
\section{Weekly Blog Posts}
\subsection{Tyler Jones's Blog}
\subsubsection{Fall Term}
\paragraph{Week 1}
This week I attended class and learned a great deal about the project and adventure to come. I submitted my top five preferences for the project and am waiting to hear back this weekend.
\paragraph{Week 2}
This week we received our projects and got to have our first meeting with our client. We will have weekly recurring meetings with our client every Thursday at 1:30 pm. I will be working with Sam Cooney and Avi Sinha for this project, and we are on the project titled, "Investment Performance Mobile Application". We will be working with Brice Lemke from HedgeServ
\paragraph{Week 3}
This week we submitted our initial proposal, and also got a better idea from our client about what the product will actually be/represent beyond just the initial problem statement 
\paragraph{Week 4}
This week we focused on our problem statement for our client. We were able to get it signed and handed in on time. There are no plans currently for the project, other than to begin thinking about the requirements document which has a rough draft due next Friday. 
\paragraph{Week 5}
This week we submitted our rough draft for our requirements document, we also met with our TA for the first time this week, Andrew, after it requiring a period of time to agree on a weekly time slot. We need to update our rough draft to be a working final copy, and send it to Brice for approval as soon as possible this week 
\paragraph{Week 6}
This week we discovered and we had it shared with us that our client will no longer be at HedgeServ. We were supposed to supply our client with our requirements document by Friday, and he didn't respond on any of the days we reached out. We had an email sent to us by one of his coworkers stating that he was no longer working there and we would have to begin working with someone else. We submitted our "final" requirements document unsigned to our TA, and will need to have it signed at a later date. We have plans to meet at our new time with our new client on Wednesdays from 12-1. We will bring him/her up to speed as soon as possible in order to keep the project moving 
\paragraph{Week 7}
This week we had our first meeting with our new clients, Edison Tsai, and Ronald Olshausen. Ronald's vision for our project differed greatly from Brice's so moving forward we are going to have to more or less go back to the drawing board and start from scratch on our requirements document. 
 
We updated Andrew, Kirsten, and Kevin all individually and informed them of the situation. Going forward we have loose deadlines on some of the documents, and can request extensions if needed. 
 
This coming week we plan on redoing our requirements document and submitting that to Edison/Ronald for review. 
\paragraph{Week 8}
This week we met with our new client for the second time and got a much better idea of what their vision is for the new project. It was very reassuring, and we received enough information to fully revise and resubmit our requirements document to our client for approval. Next week, we have our tech review due, but Kirsten gave us an extension. I plan on submitting the tech review after the break. 
\paragraph{Week 9}
This week we submitted our new requirements document and discussed it with our client in our weekly meeting. Not much was done or shared this week, and everyone in our group is going different places for thanksgiving and going to write and submit the tech review this weekend/week. 
\paragraph{Week 10}
This week, I finally got my Tech Review in after receiving an extension from Kirsten. Additionally, we began discussing the design document and received an extension on that until next Friday 12/8. We also discussed our meetings over winter break with our client, and we agreed that there wasn't a need to meet during our time off and we would reconvene in January. We also got feedback from Andrew about our requirements document, and he shared with us that our biggest flaw was missing IEEE standards, and that we need to make sure to meet them when he grades the design document and in all future documents going forward. 

\subsubsection{Winter Term}
\paragraph{Week 1}
This week I did research on the data portion of the application. Moreover, we reconnected with our client and set up a meeting for next week and every week on Wednesday at 2:30-3. Development and actual coding will start next week
\paragraph{Week 2}
This week we actively worked to get Xamarin tutorials finished, as well as made serious headway on the Alpha release of our project.  I personally did more tutorials and implemented OkHttp Client in order to retrieve sample data from our MySQL database
\paragraph{Week 3}
This week was a bit more limited on work completed than I would have liked to have seen. I had a handful of other projects for other classes as well as personal items that limited me from putting much work into the project at all.
\paragraph{Week 4}
This week I made great progress. I started the week having an enormous amount of problems with my environment and this prevented me from doing much development at all at the start of the week. From my end, I eventually got my environment configured properly and working right, and was able to get a prrof of concept working for the data flow from our application to our database. Fine tuning and compilation of mine, Avi and Sam's work comes next.
\paragraph{Week 5}
This week I made a ton of progress on the back end portion of the application. I was able to complete/develop a REST api that allows me to communicate with my MySQL database and successfully return JSON. I was also able to parse that JSON and build a simple login page. To accompany this, I successfully established key tables within our DB that will be used going forward to implement our necessary data relationships.
\paragraph{Week 6}
This week I worked on improving my existing code. I worked to further improve the queries and views that I have set up in my database in order to allow for more generalized and easy to handle JSON returns. 
Moreover, I am close to a fully functional login flow for a user. With our 3 main pages mostly in tact. I have been reading up on and trying to implement my functionality that will allow a user to login, fetch his portfolios, and access the unique data contained within those portfolios. This week has mostly been code improvements in my RestService to allow for more generalized calls to my REST API.
\paragraph{Week 7}
This week we were able to get the UI dataflow fully functional, we worked out the fetching of the portfolios, as well as even more imporvements to our API and database tables to further support our needs. 
\paragraph{Week 8}
This past week I made great progress on the project. I was able to make the proper code improvements so my login page, portfolio fetch, and investment details are all asynchronously loaded correctly. Moreover, I got involved with Xamarin Charts, which is what we are going to use to display the data in a graphical format to the user. 
 
Moreover, we received more information from our client this week regarding the business logic of our app, and moving forward we are looking to implement these business portions successfully.
\paragraph{Week 9}
This week we got the majority of our console applciation done after some discussion regarding the business logic that was discussed last week. The console application will seek to circumvent the relatively large amount of overhead that will be involved in making the proper API calls. We are able to parse data succesfully from the API and turn that into data values that we care about within our database. These values include Free Cash Flow To Equity (FCFE), Return on Assets, Return On Equity, Return On Common Equity, Debt Service Coverage Ratio, Receivables, Asset, and Inventory Turnover Ratios and finally EBIT Margin. 
All of these should be provided to the user from a company perspective. We still have a handful to calculate, and the nuances of some are more complicated than others, but generally speaking the console application is working. 
\paragraph{Week 10}
This week we are getting down to fine tuning. Our console application is coming along, and I am currently working and have been working on getting the portfolio level calculations done. These  include sharpe ratio, alpha, and r\^2. Each of these values are relatively complex in nature, so we are actively working with our client and students from OSIG to ensure we are doing it correctly as we go.  
 
We are also fine tuning some use cases and the configuration of our database so as to achieve the desired relationship within our application. These include adding purchase price of an investment, and changing how the DB is updated whenever the user does an operation in our application. 

\subsubsection{Spring Term}
\paragraph{Week 1}
This week it was shared with us that Ronald, our primary point of contact at HedgeServ, had been replaced and was no longer with the company. Moreover, Edison, our remaining contact at HedgeServ stated that it would perhaps be better moving forward to cancel our meetings. Sam, Avi and I resolved to press on with the project. In regards to the code, the console application made significant progress and our bizzare async behavior that we were seeing has also been fixed and isolated. 
\paragraph{Week 2}
This week I discovered some more nuance with the alpha vantage API and adjusted my code to account for this nuance with the occasional inability to service some API requests. Moreover, I got sharpe ratio finished, as well as alpha and expected returns. R\^2 was removed from our desired calculations, as the nuance involved with calculating r\^2 was out of the scope of this project.
\paragraph{Week 3}
This week, we put most of our finishing touches on the project. These changes were mostly cosmetic and to the UI from my end. The only remaining work that needs to be done is Avi's work regarding the asset details from the Last10k API. Regarding the rest of the term, we just have progress reports and expo to look forward to!
\paragraph{Week 4}
\paragraph{Week 5}
\paragraph{Week 6}
\paragraph{Week 7}
This week we presented our project at expo. In the interest of circumventing the heavily overloaded network that day, we pre-recorded a demo of our application and had it on a loop on a monitor at our station. Expo was a very rewarding experience, and I really enjoyed discussing our project from both a technical and financial perspective with those who were interested.
\paragraph{Week 8}
\paragraph{Week 9}
\paragraph{Week 10}

\subsection{Project Documentation}
\subsubsectioon{How does our project work?}
Our project is a cross platform mobile application that utilizes Xamarin. Xamarin is a Microsoft owned piece of development software that enabled us to develop our application for both Android and iOS with one unanimous codebase. Rather than writing our code in Java or Swift, as would be the case of typical Android or iOS applications, Xamarin enabled us to use C# for all of our development.

From a birds-eye view, there are 3 major components to our application. The first and most obvious is the user-facing application. The application seeks to allow users to track the progress of their investment portfolios, as well as see the performance of companies contained within their portfolio. The user has to ability to see individual data sets provided for companies that are listed in the DOW Jones index, as well as gain insight into their investment choices through the use of Jensen's Alpha, r\^2, and the portfolio's realized returns.

The remaining components of the application enable the user-facing front end to be useful. All of the user data for a given portfolio, as well as any company specific data are stored in a MySQL instance. Pricing data that is Incorporated into the calculation of alpha and r\^2 is retrieved from a third party free web API called Alpha Vantage. 


\subsubsection{How does one install our software, if any?}
NOT SURE WHAT TO SAY HERE. I THINK WE JUST SAY "NO SOFTWARE TO INSTALL"
\subsubsection{How does one run it?}
A user would run our application by first ensuring they have Visual Studio with Xamarin downloaded. Both pieces of software are free to download. After this, the user would simply download our .sln file and open it in a Xamarin project within Visual Studio, and begin running the emulator with a fresh build of our project.
\subsubsection{Are there any special requirements to run our software?}
Due to our project being a mobile application, a user would need to ensure that they have downloaded and installed a working Android or iOS emulator, depending on their native environment. If a user has an Android phone, Xamarin also supports direct deployment of the actual application onto the phone.
\subsubsection{Other (User Guides, API documentation, etc.)}
\subsection{Conclusions and Reflections}
\subsubsection{Tyler Jones}
\paragraph{What technical information did you learn?}
I learned a remarkable amount of technical information over the course of this project. Every single component of this project was a new piece of technology for me that I have yet to receive formal training in. Databases, C#, mobile applications, asynchronous behavior, and APIs were all my main components I interacted with, and each of them was brand new to me. I had the added benefit of the fact that I also taught myself enough about each of these concepts to meet the project goals, and I think I am much better off than my peers due to the increased retention that self-teaching offers over a class room setting.
\paragraph{What non-technical information did you learn?}
I learned that sometimes the real world is really messy and can get in the way of your goals. Repeatedly, over and over, our project took massive hits. Whether it was severe client disconnect, TA absence, or seemingly insurmountable technical barriers, it seemed that our project was doomed at times. However, I learned that facing these obstacles head on and readjusting the vision of the project when necessary is what is required in order to to complete a project that was as fragmented as ours was at times. In the real world, there are so many factors that will be out of my control as an engineer, and I will need to roll with the punches. This project has given me the opportunity to learn a great deal about myself and my work ethic when faced with such challenges.
\paragraph{What have you learned about project work?}
I have learned that project work needs good planning. I think many things were out of my control regarding the track of the project, such as consistent client issues, but others were in my control to minimize the effect that such an issue would have on the project. I learned that by starting early, and really making sure my plan and design are thought out and sound will save me countless hours as the project progresses. I also learned that project work needs time and attention. As simple as this seems, this lesson really resonated with me as the year progressed. Unlike my other course work, which comes and goes, and is only dependent on you to complete, project work comes with a variety of other challenges that require patience and time, especially in the case where you are providing a service to an external client, and not just a professor.
\paragraph{What have you learned about project management?}
Overall, the biggest thing I have learned is the importance of planning. Forgetting all of the misfortune that seemingly fell on our project, I feel as if the impact of such misfortune could have been minimized had I taken the planning process more seriously. Often, its difficult to put in hours of planning to something that you don't know how to do, as was the case with my components of this project, and with other outside factors demanding your attention, it is easy to shortcut the process or "just get it done". I have learned that a massive amount of time and worry can be spent on trying to compensate for the lack of planning, and the project and decisions made down the road may end up not being ideal due to the haste under which they were made. Planning is so crucial to any project, especially when typically your specs are not your own. If an outside, sometimes not technical, source is dictating the requirements of the project, it is necessary to truly make sure that you and the client understand the limitations of the software being used, and what exact goals you are seeking to meet.
\paragraph{What have you learned about working in teams?}
Regarding project management, the biggest thing I have learned is that, ultimately, people are going to be a real limiting factor in the course of a project. Sometimes, as an individual, no matter how well or early a plan is created, team dynamics or personality types can really play an unexpected role in the vitality of the project. If one team member slacks, or feels entitled, or feels that they have done the lion's share of the work, these feelings can result in a toxicity level that can ruin the project from the inside out. I feel its necessary with project management to get a good handle on what each member is good at, and what each member is bad at, and to be candid about such answers yourself. Distributing work and managing expectations of individual team members is necessary for the health of the project to remain in tact.
\paragraph{If you could do it all over, what would you do differently?}
If I could do this project all over, as I touched on previously, I would make sure that I knew what the project truly entailed as we went into winter term. The planning process on my end was not exhaustive enough to truly give a good sense of what was required of the project. Especially considering the nature of the components that I was responsible for being all completely new technologies, taking advantage of the allotted 10 weeks to really make sure I knew what was common practice with these technologies, how they worked, and how they could be used best to suit our needs would have saved me countless hours of stress and late nights in Winter term. Such planning mishaps as I experienced resulted in the "wrong" kind of difficult during our development process. Rather than being able to focus on the nuance of the technologies and meeting stretch goals and having a great sense of what was possible and not possible, so many hours on my end went into teaching myself how to use any of these technologies from the most basic level before I could even begin to consider thinking about how to implement them under the context of our project. This resulted in a certain kind of inefficiency and stress that I hope to avoid at all costs going forward with future projects.
\end{document}
