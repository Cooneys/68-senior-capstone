\documentclass[letterpaper,10pt,titlepage,journal,compsoc,draftclsnofoot,onecolumn]{IEEEtran}
\linespread{1}
\newcommand\tab[1][1cm]{\hspace*{#1}}
\usepackage{graphicx}                   \graphicspath{{images/}}                     
\usepackage{amssymb}                                         
\usepackage{amsmath}                                         
\usepackage{amsthm}                                          

\usepackage{alltt}                                           
\usepackage{float}
\usepackage{wrapfig}
\usepackage{color}
\usepackage{url}
\usepackage{listings}

\usepackage{balance}
\usepackage[TABBOTCAP, tight]{subfigure}
\usepackage{enumitem}
\usepackage{pstricks, pst-node}

\usepackage{geometry}
\usepackage{titling}
\geometry{textheight=8.5in, textwidth=6in}



\newcommand{\cblue}[1]{{\color{blue}#1}}

\usepackage{hyperref}
\usepackage{geometry}

\def\name{Tyler Jones}


%% The following metadata will show up in the PDF properties
\hypersetup{
  urlcolor = black,
  pdfauthor = {\name},
  pdfkeywords = {cs462 ''Senior Capstone''},
  pdftitle = {CS 462 Senior Capstone: Progress Report},
  pdfsubject = {CS462 Senior Capstone},
  pdfpagemode = UseNone
}

\title{Investment Performance Mobile App: Spring Term Progress Report}
\author{Tyler Jones, Aviral Sinha, Sam Cooney}

\begin{document}
\begin{titlingpage}
    \maketitle
	\centering{}
    \begin{abstract}
        
     The purpose of the following document is to review and discuss the current state of our investment performance mobile application as our development has begun over the last 6 weeks of Winter term. Our group will break down the struggles that we faced with the project as individuals, as well as struggles with the project overall.
        
    \end{abstract}
\end{titlingpage}

\newpage

\tableofcontents{}

\newpage

\section{Introduction to Project}

\tab Our project is an investment performance mobile application. Our main goal with this application is to provide users with a way to enter and track the progress of their investments, so they can gain unique financial insight about the quality of said investments. Typically, through a brokerage, for example, users may only see what investments they have, and they are not always given a composite performance, as is the goal with our project. We aim to give the user a metric for both their entire investment portfolio, as well as the ability to drill down into individual investments for a more localized look. At the beginning of development, we sat down as a group and went over the pieces of the project and what would be a good way to breakdown the work. Essentially, we decided that the project could be divided into 3 major components, being the API, the development of the UI, and the backend/database/console application. As will be discussed in this progress report, Avi has taken the lead on the API, Sam on UI development, and Tyler on the backend/database/console application.

\section{Sam Cooney - UI Development}
\subsection{Description of Current Status}
As of our alpha release, our UI is fully functional with all pages, buttons, and placeholders for charts/data implemented. Because of issues regarding iOS testing, the initial functional UI was built using Android libraries. Since I am new to UI design, I am using the Android test function UI to learn Xamarin and layout design. Since the functional Android UI is used to test connections to other parts of our project, I decided to create a Mock Up of our planned Beta release for the UI. We are using Xamarin.Forms for the Beta release which is a set of libraries that allows for a single codebase UI that at runtime, elements are compiled into the platform specific views. The mock-up for this beta release was created using Marvel, a mobile app mock-up site, offering free mock-up design.

This is the Portfolio View where from here users can add or remove assets from their portfolio, view vital financial data from a summarized point of all the assets currently in the portfolio. Our chart is a placeholder and will be replaced with a Google Charts designed pie chart that will be received from the web APIs.  

\begin{figure}[h]
   \caption{Portfolio View Example}
   \centering
   \includegraphics[scale=.65]{portfolio_screenshot}
\end{figure}

The Portfolio view will be where users spend majority of their time. Since we are looking to have design be built for quick, on-the-go access, the information in this view needs to be referenced quickly. This has created a challenge with balancing both simplicity but also retaining enough information that the user understands what they are looking at. Not having taken any graphical design or artistic classes, this is new and difficult for me but is proving to be an exciting task.

Our current UI functions exactly as viewed in the mock-up, what is missing is the design and the web calls. The asset view is similarly programmed. It functions very closely to the portfolio, except the view expresses the individual asset data instead of the portfolio as a whole.

\begin{figure}[h]
   \caption{Asset View Overview}
   \centering
   \includegraphics[scale=.65]{asset_screenshot}
\end{figure}

The asset view is where users will spend the second most time. From here they can view things like volatility, price changes in the last few days, return on investment since addition, and other financial data retrieved from the API and database. Much like in the last view, it is proving to be an arduous task in balancing the design.

The rest of the pages not shown here have either been fully mocked up and the functionality has been implemented. The Android functional UI was displayed in the video, as well as the rest of the Beta design. 

\subsection{What's left to do}

Our next steps are continuing development on the Beta UI release. Refactoring the Android functionality into Xamarin.Forms is proving to be a new challenge but fortunately there are lots of guides online that are extremely helpful. The portfolio view, asset view, adding new assets and portfolios all need to be implemented using Xamarin.Forms in the format of the mock-up UI. From here, we need to connect both our APIs for the charts and the financial data as well as the database storing user data - login info and portfolio/asset data. Avi and Tyler are using test UIs to plan how these connections will be made. Once these are finalized, we can begin to implement them into the Beta release.

\subsection{Problems impeding progress/Solutions}

Initially we ran into research problems in early development that lead to the functional UI design using the Android UI. First, Xamarin.Forms does not have the full functionality that each individual native UI provides, and second I don't own a Mac which prevents testing deployment on the iOS platform. Since we did not want to halt development, we began work using Android libraries only. Since then I have aquired an older iMac and have since have also begun Xamarin.Forms testing on campus where Macs are available. The decision to switch to Xamarin.Forms was a group decision based off the goal to stick to our single code-base design. This on top of the fact that we are not graphic design gurus in any sense, Xamarin.Forms will not limit us in any ways in terms of development.

\subsection{Other relevant/interesting information about project}

Button creation has turned out to be extremely difficult. Finding the perfect mix of both small and simple but still providing enough information that the user needs to know where to navigate is very challenging. YouTube and the Xamarin guides are a nice guide but creating elements that are unique to our project is still a puzzle. This was something that I had not expected when first starting this project and so it has come as a big surprise to be one of the harder parts of the work.

\section{Avi Sinha - APIs and Business Logic}

\subsection{Description of Current Status}

The project is currently right at release in terms of all the work we set out to complete when beginning development in January. The UI has been fully developed to what is comparable to other brokerage applications currently on the market like Robinhood or Merrill Edge. Like described in the last progress report mid development we made the switch from Quandl to Alpha Vantage and Last10k for our financial data. Alpha Vantage would output the specific stock prices and time series data which was used by Tyler do Portfolio level calculations like the sharpe ratio, overall returns, and then the Alpha of the portfolio. Last10k was used for more asset specific details that we need to show for each company. Currently we have our application displaying the monthly adjusted closing price in a graph provided from the AlphaVantage data and then we have 13 financial ratios showing the viability of owning that security. Last10k provided an intuitive API that allowed me to get all the necessary calculations and ratios needed to show the user an in depth look at the stock. The ratios that are sourced from Last10k are as follows: Free Cash Flow, Return on Assets, Return on Equity, Inventory Turnover, Asset Turnover, EBITDA Margin, Total Current Assets, Receivables Turnover, Net Income, Earnings Per Share, Interest Coverage, Tax Rate, and the Revenue. These values are then taken to the database server and from there the mobile application will pull everything to display on the asset details page for each stock. With the completion of all these parts of the project we have effectively finished all the main requirements that are needed to our end product successful. 

 
\subsection{What's left to do}

For what we have to do there is still some work that needs to be done with UI on the iOS side. We have made the Android version our priority the version we show at expo because of the issues we have faced with xamarin where the shared codebase is making the iOS version look a bit clunky. Once that is completed we will refine any other parts of the application where problems arise. Most of is basically complete now but if any problems arise we have between now and expo to fix them and get it ready for presentation. 
 
\subsection{Problems impeding progress/Solutions}

Our biggest problem that has come up between now and the last progress report is that our client has departed from the project. Being the second time this has happened we weren’t too surprised as from what we have gotten from this is that HedgeServ is going through a lot of restructuring. While we had gotten a new client from the company last time this happened. This time around our previous client let us know that they won’t be a replacement as the project is nearing an end. Luckily our project is essentially dictated by how well we meet our requirements the loss of the client this time around didn’t hurt development too much other than the loss of someone who could give proper financial advice on some of the business logic. We resolved this by getting advice from Kirsten who assured us that everything is fine and happens regularly. We were able to get regular advice from her regarding some parts of the project that would normally be topics of discussion with our client. Other problems included just some issues we had with the integrating everything together from the console application to the UI to the APIs that were used with everything. 

\subsection{Other relevant/interesting information about project}
Below are some relevant pieces of code from the development of the Last10k API data retrieval and how it goes from the database to the actual mobile application. 

The following code sample is from the console application where the data is pulled from Last10k's specific API call for ratios. There I parse the relevant ratios I need as out of the 30 they offer I only need around 13 of them. From here I then have it stored onto the OSU database. 

       \begin{lstlisting} 
       static async void MakeRequest()
        {
            var client = new System.Net.Http.HttpClient();
            var queryString = System.Web.HttpUtility.ParseQueryString(string.Empty);
            client.DefaultRequestHeaders.Add("Ocp-Apim-Subscription-Key",
            "ced63a48eaa34a288772c71c62da184a");

            var uri = "https://services.last10k.com/v1/company/ratios?";

            var response = await client.GetAsync(uri);
        }
    }
       
       \end{lstlisting} 
       
      This establishes the connection with the API where we specifically call the ratios that are stored in their database. Next the data is parsed in pretty straightforward way where I call each of the ratio names as such and then return their outputs. 
      
      \begin{lstlisting}
      internal class Ratios
    {
        public string Ticker { get; set; }
        public string InventoryTurnover { get; set; } 
        public string ReturnOnAssets { get; set; }
        public string ReturnOnEquity { get; set; }
        public string EBTMargin { get; set; }
        public string AssetTurnover { get; set; }
        public string TotalAssets { get; set; }
        public string ReceivablesTurnover { get; set; }
        public string NetIncome { get; set; }
        public string EarningsPerShare { get; set; }
        public string InterestCoverage { get; set; }
        public string TotalCurrentAssets { get; set; }
        public string TaxRate { get; set; }
        public string FreeCashFlow { get; set; }
        public string Revenue { get; set; }


        public override string ToString()
        {
            return $"{Ticker,20}
                     {InventoryTurnover} 
                     {ReturnOnAssets} 
                     {ReturnOnEquity} 
                     {EBTMargin} 
                     {AssetTurnover} 
                     {TotalAssets} {
                     ReceivablesTurnover} 
                     {NetIncome} 
                     {EarningsPerShare} 
                     {InterestCoverage} 
                     {TotalCurrentAssets} 
                     {TaxRate} 
                     {FreeCashFlow} 
                     {Revenue}  "; 
            }
    }
    
      \end{lstlisting} 
      
After this is all done in the console application I then take the data from the API which has been stored on our ENGR database use that to display it on our mobile application.

     \begin{lstlisting} 
      string Url = "http://web.engr.oregonstate.edu/~jonesty/api.php/Investments";

            string content = await client.GetStringAsync(Url);
            JArray investments = JArray.Parse(content);

            if (investments.Count == 0)
            {
                return null;
            }
            else
            {
                for (var i = 0; i < investments.Count; i++)
                {
                    CompanyInfo info = new CompanyInfo();
                    //tempPortfolio.Owners.Add((string)data[i]["username"]);
                    info.tickersymbol = ((string)investments[i]["tickersymbol"]);
                    info.currentprice = (double)investments[i]["currentprice"];
                    info.ReturnOnAssets = (double)investments[i]["ReturnOnAssets"];
                    info.ReturnOnEquity = (double)investments[i]["ReturnOnEquity"];
                    info.EBTMargin = (double)investments[i]["EBTMargin"];
                    info.AssetTurnover = (double)investments[i]["AssetTurnover"];
                    info.ReceivablesTurnover = (double)investments[i]["ReceivablesTurnover"];
                    info.NetIncome = (double)investments[i]["NetIncome"];
                    info.EarningsPerShare = (double)investments[i]["EarningsPerShare"];
                    info.InterestCoverage = (double)investments[i]["InterestCoverage"];
                    info.TotalCurrentAssets = (double)investments[i]["TotalCurrentAssets"];
                    info.InventoryTurnover = (double)investments[i]["InventoryTurnover"];
                    info.TaxRate = (double)investments[i]["TaxRate"];
                    info.FreeCashFlow = (double)investments[i]["FreeCashFlow"];
                    info.Revenue = (double)investments[i]["Revenue"];

                    infoList.Add(info);
     
     \end{lstlisting} 
     
     After calling the information from our database it would then be displayed using both a xaml file and xaml.cs file which contains most of the work required for both portfolio and asset level details. With the completion of this we were able to complete all the major parts of the project. 
     
     
     
     
 
    

\section{Tyler Jones - User Data and Backend}

\subsection{Description of Current Status}
 Coming into this project I have never worked with mobile applications, and I haven't taken the databases course, yet I am handling the backend for a mobile application so I have had quite the ramp up time in terms of learning how to handle my portion of the work. During fall term, it was hard to gauge if my planning and research was complete, because I didn't know what I didn't know, and if there were components that I was perhaps missing or not understanding, which definitely ended up being the case. 

Currently, however, I am working on improving the database structure. As is, it has been a significant time gate for myself to research and understand how to implement our data relationship into a relational database, as well as how to effectively interact with such data from a mobile application. The basic relationship we need to have configured for our data is as follows: Multiple users can share a portfolio and a single user can have multiple portfolios. Moreover, a portfolio can and will be comprised of multiple investments, and a certain investment can belong to more than one portfolio (multiple people can buy Apple stock). Currently, with this multiple many to many relationship structure that needs to be setup, my database structure can be seen below.

\begin{figure}[h]
\caption{Database Table Overview}
\centering
\includegraphics[width=\textwidth]{database_overview}
\end{figure}
A many to many relationship, as I have learned as I have been building my database, is best fixed by adding a 3rd joining table for the relationship. As is demonstrated by the screenshot of all the tables in my database, the naming conventions I have set up, for instance with UsersPortfolios, allows the relationship between multiple users and multiple portfolios to be achieved within the context of a relational database like MySQL. The same goes for PortfoliosInvestments when showing the multiple investments that make up a portfolio. Without going into too much detail, how the process will work, and is working currently, will be that when a user adds a new portfolio, our application will make a call to the API and push the information about that portfolio, including portfolio name and total amount invested in the portfolio into the Porfolios table, as well as insert a new entry into the UsersPortfolios table that specifies who owns that new portfolio, which could be one or multiple users. All of the data that will be queried from the database is from the pictured UsersPortfoliosView, which places all of the portfolios and their owners into a single table, or technically a view in this case, that is able to be queried by our API effectively. This same strategy is applied in the PortfoliosInvestments table, and will also have a corresponding view that allows for effective queries.

Another thing that I have been working on is the interaction of this data with my database. Coming into this term, I wasn’t aware what an API was or what it was used for, but I ended up having to build and use one for our application. These first few weeks of the term have been dedicated to research, as well as constructing our REST web API on our web.engr hosted OSU site. This REST web API is based in php and returns JSON for our C\# based application to parse and display for the user.  The API is generalized so that calls are made to it in the form of “…/api.php/{TableName}”, the output from such a URL is JSON of all the content of the desired table. I felt that making my API simple enough so that calls to it can be generalized and simply return all the content from the table allows me to do my complexities involving the data  through the database and client side, rather than getting bogged down in making an overly complex API. 

I was overall in charge of two more major components. The first of these components was the API that would communicate with our database. This was a simple php based web api that is housed on my engr web space that permits the user to either return the contents of a table in json format, or upload values to the database when needed within the mobile application. As stated in my previous progress report, I had never worked with API's coming into this term, so having to teach myself the concept was a significant amount of ramp up time. This REST web API is based in php and returns JSON for our C\# based application to parse and display for the user.  The API is generalized so that calls are made to it in the form of “…/api.php/{TableName}”, the output from such a URL is JSON of all the content of the desired table. I felt that making my API simple enough so that calls to it can be generalized and simply return all the content from the table allows me to do my complexities involving the data  through the database and client side, rather than getting bogged down in making an overly complex API. However, reflecting on what I know now, I would navigate away from this API. It ended up being sufficient for our needs for the project, but I have learned a lot since we first used it for our implementation, and I know of many different or better things I would want to attempt with it going forward.

The second major component of this project that was mine to handle was the console application that dynamically calculates user portfolio ratios. This application is a C\# based .Net framework application that communicates with the Alpha Vantage API. This API provides us with our stock pricing data. The portfolio level ratios that we care about are Sharpe Ratio, Alpha, and returns. The main intent of this application was to provide us with a way of conveniently circumventing the significant overhead that goes into each of these calculations. The console application makes an API call for the given portfolios that are in the database, and then runs a considerable amount of calculations on both the portfolios and the pricing data for the companies that are in the portfolio to ultimately reupload the listed ratios to the database again. 

\subsection{What's left to do}

What I have left to do and where the project as a whole is going are two different things. When composing the final product and putting all of our pieces together, my backend portions will mostly be plugged in to Sam's work with the UI. The work I personally have left to do on the backend is registering a new user, adding another user to an existing portfolio, figuring out how to configure the database to handle specific numbers of shares being associated with a portfolio, and lastly, adding hashing functionality to our users' passwords, as right now they are being stored just as plain text. As far as the application as a whole is concerned, the biggest piece that we need working now are our calls to the financial API, Last10k and Alpha Vantage, that will use fetched data in order to actually allow our app to do what it was intended to do. This will be a challenge not so much from a technical standpoint, but rather from a business standpoint, and will require verification from our client going forward that we are using the correct data, and generating and showing the user their investment performance properly. Overall, as we have gotten a much better sense over these first few weeks about what degree of work is going to be required for all the components, as well as a more complete and solid picture of how our app is going to work as a complete unit, we feel confident moving forward about the state of the project.

Currently, as of today, there is not much left to do regarding the user data portions of our application. The database has been configured to our standards, and is properly serving the purpose we need for within our mobile app.

\subsection{Problems impeding progress/Solutions}

One bottleneck and problem, as I previously mentioned, that I have experienced so far is being thrown into many components of this project that are entirely new. The paradigm of mobile development, C\# which our app is written in, MySQL for the database, php for the API, and JSON parsing within the app are all completely new tools that I am having to both learn and effectively use all without having any prior knowledge of.

The biggest problem that we have had thus far other than personal bottlenecks with having to ramp up our own knowledge of the technical portions is client issues. At the start of this term, Ron and Edison, our clients, sort of bailed out on the project. Ron was fired from the company, and Edison said it was perhaps best to stop our calls for the time being. This wasn't so much a wrench in the project, as close to all of the work had been done by the time this happened, but was rather just a morale blow that has made the project feel aimless since it now doesn't really have a client that it's being produced for.

\subsection{Other relevant/interesting information about project}
Regarding the work breakdown of the project, Sam, Avi, and I are all individually working on components of the application, so although we haven't put them all together yet, the act of doing so will be trivial. All of us are new to mobile development, as well as Xamarin, so we figured this work flow would be much more efficient while we more or less figure out how to get each of our necessary components working. Where my portion of the work currently stands is mostly with allowing the user to communicate with the database within the application. Currently, users have the ability to log-in, albeit insecurely at the moment, load their portfolios, and load the investments contained within said portfolios. Furthermore, Xamarin has proved to actually be an incredibly interesting tool for our development. Developing a cross-platform application in C\# feels a bit like magic at times, as neither iOS or Android use C\# natively. It has been interesting developing from the standpoint of making an app that is as cross-compatible as possible, and at times this means sacrificing some of the unique and powerful qualities that iOS and Android both uniquely offer. 


\section{Conclusion}
The project so far has made nice progress and we are beginning to see the fruits of our labor. As we start bringing the individual parts of our project together for the final release, our application will feel like the real deal. As of now we are now focusing on perfecting our current product so it can be polished and ready for expo in the coming weeks. 


\end{document}
